% --------------------------------------------------------------------------
% Template by Leonard Mandla Mbuli (mail@mandla.me) & Adam Pantanowitz (adam.pantanowitz@wits.ac.za)
% School of Electrical and Information Engineering
% University of the Witwatersrand
%
% compiled with (this is assuming you are using minted):
% pdflatex -shell-escape main

% REQUIREMENTS:
% fullpage - for fullpages
% graphicx - pictures
% xcolor   - pretty colours easier naming
% minted   - for code highlighting
% url      - for urls in the the text and in references
%
% More information on minted can be found in the doc/minted.pdf
%
% created: 20 December 2011
% updated: 10 May 2014
% ---------------------------------------------------------------------------
\documentclass[11pt]{article}
\usepackage[margin=2.5cm]{geometry}
\usepackage[usenames,dvipsnames]{xcolor, colortbl}
\usepackage{graphicx}
\usepackage{url}
\usepackage{titlesec}
\usepackage{booktabs}
\usepackage{ragged2e}
\usepackage{lipsum}
\usepackage{listings}
\lstset{
  basicstyle=\fontfamily{lmvtt}\selectfont\footnotesize\color{black},
  columns=fullflexible,
}
\usepackage{multicol}
\setlength{\columnsep}{1cm}

\definecolor{myblue}{rgb}{0.0, 0.0, 0.5} % for the section headings and title
\definecolor{mylightgrey}{gray}{0.9} % for the tables

\titleformat{\section}
{\color{myblue}\normalfont\Large\bfseries}
{\color{myblue}\thesection}{0.5em}{}
%\renewcommand{\familydefault}{\sfdefault}

\begin{document}
\pagestyle{empty}
\begin{minipage}{0.18\textwidth}
    \includegraphics[width=\textwidth,height=\textwidth]{eie.png}
\end{minipage}
\begin{minipage}{0.8\textwidth}
    \centering
    \textbf{\Large School of Electrical and Information Engineering}\\
    {\large University of the Witwatersrand, Johannesburg}\\
    {\small Private Bag 3, 2050, Johannesburg, South Africa}\\
\end{minipage}
\vspace{.3cm}\\
\colorbox{myblue}{\begin{minipage}{0.98\textwidth}
        \begin{center}
            \textcolor{white}{\textbf{\Large{ELEN4017: Project - 2017}}\\
                \textbf{Networks:  Email Application System}\\
                May 2017
            }
        \end{center}
    \end{minipage}
}
\\
\vspace{6cm}
\begin{center}
{\Huge Benjamin Thomas (Ben)} \\
{\Large 545787} \\
{\large 545787@students.wits.ac.za}
\break{}
\break{}
{\Huge Benjamin Shear (Benji)} \\
{\Large 749992} \\
{\large 749992@students.wits.ac.za}
\end{center}
\vspace{5cm}
\begin{abstract}
\justify
    
\end{abstract}
\newpage
\begin{multicols}{2}
\section{Introduction}
The Electronic mail, or e-mail, was created in the early 1970's and has since grown to become an area of technology which people interact with on a daily basis. It forms a significant area in human lives for people to communicate and interact with one another and effectively demonstrates an interaction between the application layer and transport layer. Although emails are easily accessible now from all over the world, it was not always like this. The first email services made use of a protocol known as the Post Office Protocol (POP) which allowed users to access their emails from a single personal computer. Once an email had been fetched from the server, it was stored on a personal computer and then deleted off the server. This meant that people would essentially need a personal computer dedicated to their own emails, so that if need they could always have access to their already read emails. With the change in technology, and the introduction of more portable devices such as laptops and cell-phones, the Instant Message Access Protocol (IMAP) was developed. IMAP allowed users to access their emails from any device. It did so by having a central server which would store all the users emails, and when the user wished to view an email, a copy of the email would be sent to the users device leaving the original email on the server. The server only deletes the email if the user requests to delete the specific email.\\
This report will start by discussing the implemented and missed features of email application developed in the project. Following this, the protocols used will be discussed as well as the division of work. WIreshark will then be used to demonstrate the network interaction. A critical analysis of the application will follow to discuss the success and failure of the email application and finally the report will be concluded. 
\section{Implemented System}
The system works as a basic email application and makes use of a simple GUI to help the user easily navigate through the application. The system is able to allow a user to connect to the required Simple Mail Transfer Protocol (SMTP) server and log in using the user's user-name and password. The user then has the option of either receiving emails, or sending an email. If the user wishes to receive emails, the user is then prompted to select the protocol to use to receive emails. The user can either select the IMAP protocol, or the user can select the POP3 protocol. Certain requests (such as LIST for POP3 and FETCH for IMAP) require the user to enter arguments into the functions so that the application knows which email to list/fetch. Under such circumstances, the user is then prompted to enter the argument into a text field so as to inform the server of what it needs to list/fetch. Although not ideal, this means that the user needs some understanding of the IMAP and POP protocols to make use of the application.
\subsection{Implemented features}
The application has the ability to carry out the following features:
\begin{itemize}
  \item Log-in using SMTP protocol
  \item Receive and delete email via IMAP4 protocol
  \item Receive and delete email via POP3 protocol
  \item Send email via SMTP protocol
\end{itemize}
\subsection{Missed features}
The application is able to effectively send and receive emails, however the application is only able to handle emails that consist of text. Due to time constraints, the ability to send and receive images was not implemented.
\section{Protocols}
The application makes use of three main protocols, namely the SMTP, IMAP and POP3 protocol. Each protocol is unique and functions in a different way. The SMTP protocol is used to send emails where as the IMAP and POP protocols are used to receive emails.
\subsection{SMTP}
Whenever an email is sent, the computer connects to an email service’s mail server (a centralized computer which manages manages emails). The SMTP server is the server responsible for sending emails. One SMTP server can pass on the mail to another SMTP server and relay the mail to the destination through several hops. When an email is sent, the email client connects to the SMTP server of the sender’s email service and transmits the address of the sender, the address of the recipient and the content of the message. The SMTP server obtains the domain name from the recipient’s email ID, and uses it to determine the location of the recipient's SMTP server. The process is a lot simpler if the recipient’s and sender share the same email ID as the SMTP server can merely transfer the mail to its local outgoing mail server (POP3 or IMAP) instead. The SMTP server then contacts the DNS Server to obtain the recipients Internet Protocol (IP) address and the the DNS server sends back the address to the SMTP server. The SMTP server then proceeds to hand over the email to the SMTP server of the recipient’s email service. This SMTP server then checks and confirms that the mail addressed to the receiver belongs to it, and finally hands it over either the POP3 server or the IMAP server for the mail to be received.
\subsubsection{Command}
\begin{itemize}
  \item EHLO localhost - Sends the EHLO command as per SMTP protocol
  \item AUTH LOGIN - Sends the request to login. The server then responds, requesting client to send username and password
  \item MAIL FROM: [argument] - Requests the senders mail ID
  \item RCPT TO: [argument] - Requests the receivers mail ID
  \item DATA - Requests the data to be sent. The server interprets a full-stop on an empty line as the token that informs the server when the client has finished sending data.
  \item NOOP - 
  \item VRFY [argument] - Verifies if a user exists on the server
  \item RSET - 
  \item QUIT - 
\end{itemize}
\subsubsection{Reply}
\begin{itemize}
  \item EHLO localhost - 
  \item AUTH LOGIN - '235 2.7.0 Accepted' if the username and password match a username and password in the SMTP server. '535-5.7.8 Username and Password not accepted' if either the username or password do not match a username and password in the SMTP server.
  \item MAIL FROM: [argument] - '250 OK'
  \item RCPT TO: [argument] - '250 OK' if the recipients mail ID exists or '550 No such user here' if it does not.
  \item DATA - 
  \item NOOP - '250 OK'
  \item VRFY [argument] - 
  \item RSET - 
  \item QUIT - 
\end{itemize}
\subsection{POP3}
Post Office Protocol (POP3) servers are the servers responsible for receiving mails. POP3 servers have mail accounts which are mapped to a username-password combination which allow authentication of the user attempting to receive mail. Once the message is handed over to the POP3 server, it is kept and stored in the mail account till the recipient logs in and checks the mail. This happens when an email client connects to the POP3 server and authenticates itself using the username-password combination. The email client then tells the POP3 server to allow the email client to download email. Once the mail has been downloaded to the local machine, the POP3 server deletes the mail off the server. This means that once the mail has been downloaded, it no longer exists on the POP3 server and can only be accessed from the device that it was downloaded onto.

\subsubsection{Command}
\begin{itemize}
  \item USER [argument] - This allows the user to pass their user-name to the POP3 server for authentication
  \item PASS [argument] - This allows the user to pass their password to the POP3 server for authentication against the user-name already entered
  \item STAT - This allows the user to view how many emails are in the inbox as well as the size (number of bytes) of the inbox
  \item LIST [argument] - This allows the user to view the size of a specific email. If no argument is passed, the server returns the size of each email in the inbox.
  \item RETR [argument] - This allows the user to retrieve a specific email corresponding to the argument passed in.
  \item NOOP - This does nothing.
  \item DELE [argument] - This marks a user specified email for deletion.
  \item RSET - This clears all emails that have been marked for deletion.
  \item QUIT - This closes the connection.
\end{itemize}
\subsubsection{Reply}
\begin{itemize}
  \item USER [argument] - '+OK Send PASS'
  \item PASS [argument] - '+OK Welcome' if the password matches the user-name, '-ERR [AUTH] Username and password not accepted' if it does not.
  \item STAT - '+OK [number of messages in inbox] [size (in bytes) of inbox]'
  \item LIST [argument] - '+OK [message number] [size (in bytes) of message]' or '-ERR Message number out of range' if the argument passed is larger than the number of messages on the server.
  \item RETR [argument] - The server will either reply with the content of the message or '-ERR Message number out of range' if the argument passed is larger than the number of messages on the server.
  \item NOOP - '+OK'
  \item DELE [argument] - '+OK Message marked for deletion' or '-ERR Message number out of range' if the argument passed is larger than the number of messages on the server.
  \item RSET - '+OK'
  \item QUIT - '+OK Farewell'
\end{itemize}

\subsection{IMAP}
Due to the shortcoming of POP3, where the email is deleted off the server once it has been downloaded, IMAP was developed and introduced. IMAP4 (Internet Message Access Protocol version 4) works in an identical manner to POP3, however it retains a copy of the emails on the server allowing the user to access their e-mail from any location with an internet connection
\subsubsection{Command}
\begin{itemize}
  \item LOGIN [username] [password] - 
  \item SELECT INBOX - 
  \item CAPABILITY - 
  \item LIST INBOX - 
  \item DELETE inbox - 
  \item FETCH [argument] - 
  \item SUBSCRIBE [argument] - 
  \item UNSUBSCRIBE [argument] - 
  \item CLOSE - 
  \item COPY - 
  \item LOGOUT - 
\end{itemize}
\subsubsection{Reply}
\begin{itemize}
  \item LOGIN [username] [password] - 
  \item SELECT INBOX - 
  \item CAPABILITY - 
  \item LIST INBOX - 
  \item DELETE inbox - 
  \item FETCH [argument] - 
  \item SUBSCRIBE [argument] - 
  \item UNSUBSCRIBE [argument] - 
  \item CLOSE - 
  \item COPY - 
  \item LOGOUT - 
\end{itemize}
\section{Detailed feature implementation}
Figure 1 of Appendix A shows the screen presented to the user upon running the application. The left half of the screen represents the receiving half of the screen and the right half of the screen represents the sending half of the screen. The user is first required to enter their user-name and password into the 'Username" and 'Password' fields, before clicking the 'Login' button. A pop-up message appears indicating if the user-name or password are correct or incorrect. If correct, the program logs into the corresponding SMTP server.
\subsection{Sending mail}
As mentioned earlier, the right half of the screen represents the sending half of the screen and will be discussed in this section. Seeing that the user has already logged into the SMTP server, the user simply needs to type in the recipients email ID and the content of the email in order to send an email. Figure 3 shows the user networks4017tester@gmail.com logged in, sending an email to benjaminthomassa@gmail.com with the content of the message being some dummy text. If the message is able to send successfully, a pop-up message is displayed informing the user that the message has been successfully sent as shown by Figure 3.
\subsection{Receiving mail}
As mentioned earlier, the left half of the screen represents the receiving half of the screen and will be discussed in this section. If the user wishes to receive mail, then the user needs to either select the IMAP or the POP3 check box. Upon doing so, the program will then either log into the IMAP or POP3 server, depending on which the user selected. This is shown by Figure 4 and 5, where the IMAP and POP3 servers were logged into respectively. If the user opted to log into the IMAP server, the drop-down box will contain items that correspond to the IMAP protocols and if the user opted to log into the POP3 server, the drop-down box will contain items that correspond to the POP3 protocols. Certain protocols (such as the POP3 RETR protocol) require the user to input additional data, in which case the user needs to enter the data into the 'Command' box. Once all the required fields have been selected or filled, the user can then hit the 'Send Command' button, which will send the command that the user has chosen, to the corresponding server (IMAP or POP3). The interaction between the client and server is displayed in the Text Area at the bottom half of the receiving half of the screen. Figure 6 shows the Text Area being filled when a POP3 RETR protocol is run. The dummy text that was sent earlier can be seen, showing that the program is able to both send and receieve emails.
\section{Division of work}
The work was divided as follows:\\
\break{}
\begin{tabular}{ |p{3cm}||p{1.5cm}|p{1.5cm}|}
 \hline
 \multicolumn{3}{|c|}{Percentage Division Of Work} \\
 \hline
  & Ben & Benji\\
 \hline
 SMTP & 0 & 100\\
 IMAP & 0 & 100\\
 POP & 100 & 0\\
 Multithreading & 0 & 100\\
 GUI & 100 & 0\\
 Report & 80 & 20\\
 \hline
\end{tabular}

\section{Results}

\section{Code structure}
The code was divided into three main separations. The first dealt with everything happening on the client side, the second with everything that happened on the server side, and the final division dealt with the code responsible for the GUI and linking both the server and client interactions with the GUI. This section will discuss the class structure within each separation, as well as how the methods operate.

\subsection{Server}
The server division was subdivided into separate sections according to the functionality of each section. This consisted of an IMAPserver, a SMTPserver, a POP3server and a ServerHandler.
\subsubsection{ServerHandler}
The ServerHandler file contains classes for handling server interactions at the TCP level. The classes manage setting up threads of sockets associated with each process. Once a TCP connection has been accepted by the either SMTP, IMAP or POP server sockets, the thread classes are run. The socket and address of the TCP socket assigned are retrieved and an instance of the server is created ensuring that each thread has its own instance of the server. The connection is then kept open until it has either timed out, or it is terminated by the user.\\
There are also three separate classes that are responsible for setting up the sockets listening for the SMTP, POP3 and IMAP services. For SMTP the program listens on port 465, for IMAP the program listens on port 993 and lastly for POP3 the program listens on port 995.\\
\subsubsection{SMTPserver}
The SMTPserver file is responsible for the interaction between the client side and the SMTP server by relaying messages received from client to another SMTP server. The server waits for a command from the client and if the command is a recognized SMTP command, relays it to the appropriate SMTP server (belonging to the client). Each thread for each individual client connection will have its own instance of SMTPserver object.
\subsubsection{IMAPserver}
The IMAP server is responsible for handling IMAP requests from the client. It processes and responds to the client according to IMAP protocol. In order to allow for threading, each connection will have its own IMAPserver object running in its own server thread. The server has the ability to login, select specific emails and fetch the information contained in that email. The server determines if the request from the client is a valid IMAP protocol request, and if so responds accordingly.
\subsubsection{POP3server}
The POP server is responsible for handling POP requests from the client. It processes and responds to the client according to POP protocol. In order to allow for threading, each connection will have its own POP3server object running in its own server thread. The server has the ability to login using the clients username and password, as well as listing, retrieving and deleting emails. The server determines if the request from the client is a valid POP3 protocol request, and if so responds accordingly.
\subsection{Client}
The client division was subdivided into separate sections according to the functionality of each section. This consisted of a client\_sockets, a SMTPclient, a POP3client and a IMAPclient.
\subsubsection{client\_sockets}
The client\_sockets file contains 3 Classes which allocate TCP sockets on the client side for the SMTP, POP3 and IMAP services. This file is responsible for sending and receiving messages to and from the mentioned servers through TCP streams. For SMTP, the program allocates port 465 for the TCP connection, For IMAP, the program allocates port 993 for the TCP connection and for POP3, the program allocates port 995 for the TCP connection.
\subsubsection{SMTPclient}
The SMTP client class is responsible for interacting with the SMTP server. The class creates a client socket object to establish a connection with the SMTP server socket. The class has to follow all SMTP protocols, such as sending an EHLO command, before authentication can occur. The class has the primary ability to authenticate the user's user-name and password with the SMTP server as well as to send data from a sender to a receiver. The function authenticate() works by first encoding the user's user-name and password. A 'AUTH LOGIN' message is then sent to the SMTP server so that the server knows to expect the encoded user-name and password from the client.
\subsubsection{IMAPclient}
The IMAP client class is responsible for interacting with the IMAP server. The class creates a client socket object to establish a connection with the IMAP server socket. The class has the primary ability to authenticate the user's user-name and password with the IMAP server as well as to send data from a sender to a receiver. The function LOGIN() works by sending the user's user-name and password to the IMAP server, preceded by the word 'login'. If authenticated, the client then has the ability to carry out the LIST, DELETE, LOGOUT, SELECT and FETCH functions which are all self-explanatory.
\subsubsection{POP3client}
The POP3 client class is responsible for interacting with the POP server. The class creates a client socket object to establish a connection with the POP3 server socket. The class has to follow all POP3 protocols, such as first sending an USER command, with the client's user-name passed in as a parameter. The server then requests a password from the client before validating the user-name and password. The client sends the password through the PASS function, with the password being passed in as a parameter. Once authenticated, the client then once again has the ability to carry out the LIST, DELETE, RETRIEVE, RESET and QUIT functions. The only function which needs clarification is the LIST function, which has the option to either pass in an argument or not to. If an argument is passed in, then the server will list that specific email (the size of the email in bytes), if not then the server will list every single email belonging to the client with each corresponding email size (in bytes).
\subsection{Graphical User Interface}
The Graphical User Interface division was subdivided into separate sections according to the functionality of each section. This consisted of a GUI, a SMTPfunctions, a IMAPfunctions and a POP3functions file. The GUI file is responsible for presenting an easy to use graphical interface and each of the function files are responsible for linking the SMTP, IMAP and POP3 servers to the GUI.
\subsubsection{GUI}
The GUI file makes use of the Tkinter package to form the graphical user interface. The GUI uses an array of Tkinter objects such as buttons, text boxes, check boxes, combo boxes and labels to form the GUI that is visible to the user.
\subsubsection{SMTPfunctions}
The SMTPfunctions file only contains two functions. The first, authenticate(), is responsible for connecting to the SMTPserver file and authenticating the user with the credentials stored on the SMTP server. The second function, sendSMTPMessage(), passes the user's email ID, the recipients email ID and the data to be sent to the SMTPserver file to request the SMTP server to send the email to the desired person.
\subsubsection{IMAPfunctions}
The IMAPfunctions file connects to the IMAPserver to authenticate the user.
\subsubsection{POP3functions}
Like the IMAPfunctions file, the POP3functions file also connects to the designated server file to authenticate the user. The file is also responsible for interpreting the easy to read user options (Retrieve, Delete etc.), into the required POP3 protocols (RETR, DELE etc.).
\section{Conclusion}

\end{multicols}
\newpage
\begin{thebibliography}{9}

\end{thebibliography}
\newpage
\appendix
\begin{figure}[h]
\centering
\caption{Main Screen}
\includegraphics[scale=0.7]{MainSCreen.png}
\end{figure}

\begin{figure}[h]
\centering
\caption{Sending Message Screen}
\includegraphics[scale=0.7]{sendingMessage.png}
\end{figure}

\begin{figure}[h]
\centering
\caption{Message Sent Screen}
\includegraphics[scale=0.7]{messageSent.png}
\end{figure}

\begin{figure}[h]
\centering
\caption{IMAP Logged In Screen}
\includegraphics[scale=0.7]{IMAPloggedIn.png}
\end{figure}

\begin{figure}[h]
\centering
\caption{POP3 Logged In Screen}
\includegraphics[scale=0.7]{POP3loggedIN.png}
\end{figure}

\begin{figure}[h]
\centering
\caption{POP3 Retrieve Message Screen}
\includegraphics[scale=0.45]{receivedEmail.png}
\end{figure}
\end{document}

